
% pdflatex book template
% this file is part of the kdissert templates
% http://freehackers.org/~tnagy/kdissert/index.html
% created by Thomas Nagy <tnagy2^8@yahoo.fr> in 2004
%
% permission is granted to redistribute copies and
% derivated work of this file without restrictions
%

% main latex includes
\documentclass[a4paper,12pt]{report}
%\usepackage[T1]{fontenc} % on some systems T1 looks ugly
%\usepackage[cyr]{aeguill}

\usepackage[pdftex]{graphicx}
\DeclareGraphicsExtensions{.png,.pdf,.jpg,.jpeg}

\usepackage[pdftex,
bookmarks=true,
bookmarksnumbered=true,
pdfpagemode=None,
pdfstartview=FitH,
pdfpagelayout=SinglePage,
colorlinks=true,
urlcolor=magenta,
pdfborder={0 0 0}
]{hyperref}

%%%%%%%%%%%%%%%%%%%%%%%%%%%%%%%%%%%%%%%%%%%%%%%%%%%
%interesting settings for the page layout

%\setlength{\hoffset}{-18pt}
%\setlength{\oddsidemargin}{0pt}  % left margin, odd pages
%\setlength{\evensidemargin}{9pt}  % left margin, even pages
%\setlength{\marginparwidth}{54pt} 
%\setlength{\textwidth}{481pt}  % set the text width to about 17cm
%\setlength{\marginparsep}{7pt}
%\setlength{\topmargin}{0pt}  % no margin on top
%\setlength{\headheight}{13pt}
%\setlength{\headsep}{10pt} 
%\setlength{\footskip}{27pt}
%\setlength{\textheight}{700pt}

%%%%%%%%%%%%%%%%%%%%%%%%%%%%%%%%%%%%%%%%%%%%%%%%%%%
%% miscalleneous packages

% \usepackage[frenchb]{babel} % use this if you are French
% \usepackage{verbatim} % include eg: source code easily
% \usepackage[utf8]{inputenc} or \usepackage[latin1]{inputenc}

% If you are on Fedora Core you may have problems with accents
% in this case replace the accentuated characters like this :
%
%  é -> \'e
%  è -> \`e
%  ê -> \^e and so on
%
% To automate the replacement you can make yourself a script
% here is a handy command-line for doing this :
% perl -pi -e "s/é/\\\'e/g" main.tex

\usepackage{amsmath}
% math extension - one probably wants to use symbols like '[' (written as '$[$')
% or $\epsilon$

% see mymacros.sty - print nice chapter headers

\makeatletter
\def\@makechapterhead#1{%
  \vspace*{50\p@}%
  {\parindent \z@ \raggedright \normalfont
    \interlinepenalty\@M
    \ifnum \c@secnumdepth >\m@ne
        \Huge\bfseries \thechapter\quad
    \fi
    \Huge \bfseries #1\par\nobreak
    \vskip 40\p@
  }}

\def\@makeschapterhead#1{%
  \vspace*{50\p@}%
  {\parindent \z@ \raggedright
    \normalfont
    \interlinepenalty\@M
    \Huge \bfseries  #1\par\nobreak
    \vskip 40\p@
  }}
\makeatother



%%%%%%%%%%%%%%%%%%%%%%%%%%%%%%%%%%%%%%%%%%%%%%%%%%%

% headers and footers for your document
\usepackage{fancyhdr}

% count the number of pages for display on footer
\usepackage{lastpage}

% width of the line for headers and footers
\renewcommand{\headrulewidth}{0.5pt}
\renewcommand{\footrulewidth}{0.5pt}
%\addtolength{\headwidth}{\marginparsep}

% uncomment the following lines for headers
% the following command need picture files : "logo-school.jpg" 
% and "logo-company.jpg" in the project directory
%
%        \lhead{\sl \includegraphics[height=1.1cm]{logo-school}}
%        \chead{}
%        \rhead{\sl \includegraphics[height=1.2cm]{logo-company}}


\usepackage{ucs}
\usepackage[utf8]{inputenc}


%%%%%%%%%%%%%%%%%%%%%%%%%%%%%%%%%%%%%%%%
%document headers and footers
\lhead{}
\chead{}
\rhead{}
\lfoot{The Yzis Development team}
\cfoot{\thepage/\pageref{LastPage}}
\rfoot{ \today } % -> \rfoot{\number\month/\number\day/\number\year} 
\pagestyle{fancyplain}



\title{The Yzis Developer Handbook}
\date{\today}
\author{The Yzis Development team}

\begin{document}
\maketitle
\tableofcontents

% 


\chapter{Introduction}

This handbook is aimed at developers wanting to work on Yzis. We will
explain here how to dig into the Yzis code, architecture details and
explanations about the tools we use.


\chapter{Technologies}
Yzis is not doing all of its own. Some external tools are used to help
build yzis. Yzis is built on top of some other software components. We describe
all of those in this chapter.

\section{cmake}
cmake is a replacement for the autotools ( autoconf/automake and such). We
decided to switch to cmake because the autotools are evil, and, moreover,
KDE officially switched to cmake. Yzis is quite close to kde, and follow
KDE decisions on such topic.
You can find more information about cmake on http://www.cmake.org

\section{subversion}
Subversion is a Software Configuration Management(SCM), or Source Control,
or Revision Control, or Version Control System (VCS). That is, subversion
is used to handle source files. It keeps the history of the changes to the
file and help the team to share the source.
Check out the subversion book to know more about this tool.

Useful links~:
\begin{itemize}
\item http://en.wikipedia.org/wiki/SCM
\item http://en.wikipedia.org/wiki/Revision\_control
\item http://subversion.tigris.org/
\item http://svnbook.red-bean.com/
\end{itemize}

\section{Doxygen}
Doxygen is a tool which generates documentation from text inserted in the
code itself. This typically used to describe APIs.
See http://www.doxygen.org/

\section{Qt}
Yzis is built on top of the \emph{Qt} C++ library, which provides some
generic useful classes and some optional GUI widgets. \emph{Qt} is most
widely known as the basis for \emph{KDE}.
See http://www.trolltech.com for more details.

\section{Lua}
Lua is a small footprint script language designed to be easily embedded in
applications. Yzis uses \emph{lua} for its main scripting language. See
http://www.lua.org.

\section{\TeX}
We use {\TeX} for documentation. You need a rather modern {\TeX} distribution.
For example, this is known to work with TeTeX 3.0 and not with TeTeX 2.0.2.

\chapter{Startup guide}
This chapter explains how to get a fully usable environment to play with
Yzis source.

First you need to get the source. Of course, you could use the latest
tarball released, but it's probably outdated already. Moreover, you will
not be able to follow the changes made by the Yzis team.
Better use a svn checkout~:

\begin{verbatim}
svn co svn://svn.freenux.org/yzis/trunk yzis-trunk
cd yzis-trunk
mkdir build
cd build
cmake ../ -DCMAKE\_INSTALL\_PREFIX=/opt/kde4 -DCMAKE\_BUILD\_TYPE=debugfull
make
sudo make install
\end{verbatim}

You can use \verb+ccmake ../+ instead of the \verb+cmake+ line to get an
interface where you can configure variables.
Now, you should have the latest yzis compiled and installed.

If you intend to develop, it's probably a good idea to have the source
documentation available~: (from within the main directory \verb+yzis-trunk+)

\begin{verbatim}
doxygen yzis.doxy
\end{verbatim}

Then open a browser on the page \verb+yzis-trunk/apidoc/index.html+ and you
have the full API documented. PLEASE update this documentation whenever you
can while coding : document things you add, complete existing
documentation, fix documentation when you find an error..

Before digging into the source, you should probably read the architecture
overview. And then.. go and read the source!

\chapter{Architecture overview}

\section{Naming convention}
It's not as if I still perfectly agree with the naming convention, but
until we possibly change it, it needs to be respected for coherency.
Classes in the core \verb+libyzis+ are prefixed with \emph{YZ} like in 
\verb+YZBuffer+ or \verb+YZView+.
Classes in client specific code have yet another letter prefixed. The
letter are \emph{G}, \emph{Q}, \emph{K}, \emph{N} for Gtk, Qt, KDE
and ncurses clients, respectively.

\section{Source code tree}
Here's a brief description of yzis source tree :
\begin{verbatim}
yzis
|-- apidoc              # generated by doxygen, html API documentation
|-- build               # cmake stores all generated files there
|-- cmake               # files used by cmake
|-- debian              # files used for debian packaging
|-- doc                 # documentation (this handbook)
|-- gyzis               # first shot at gtk client, obsolate/unmaintained
|-- kyzis               # kde client and kde embedable component
|-- libyzis             # Yzis core, static lib shared by all clients
|-- nyzis               # ncurses client
|-- qyzis               # Qt only client
|-- scripts             # (lua) scripts installed with yzis
|-- syntax              # syntax highlighting scripts
|-- tests               # unit testing framework
\-- translations        # translations files
\end{verbatim}

So, code-wise, you basically have a common core, found in the dir
\verb+libyzis+, which is a static library. It contains all GUI independant
code. Then, each client has its own directory, with the GUI dependant code,
which links against \verb+libyzis+ to create the executable.

As of december 2006, the purpose is to first make the \emph{ncurses} client
and the \emph{Qt only} client available, then the \emph{KDE} client and
embedable component, and then others.

\section{Which part handles what ?}

The \verb+libyzis+ core handles files (loading/modifying/saving), commands
interpretation, scripts interpretation (lua, highlighting), and lot of
common stuff for rendering.
It has some abstract classes about GUI, which provide APIs and basic stuff.
Most often client will inherit those classes.
It also contains some generic code (debug).

An important point is the event loop handling. The \emph{client} is
responsible for handling the event loop. The \emph{client} code will do
call to \verb+libyzis+.

\section{Core classes}

We describe here the few most important classes in \emph{Yzis}, and how
they interact.

\verb+YZBuffer+

Being a file editor, \emph{Yzis} code is centered around a class handling
files. They are either empty file (like when you start "nyzis"), or a file on
the filesystem ("nyzis foo.c").
This class is responsible for loading, saving, and modifying the file
(\verb+insertChar()+ for example). It keeps track of all views on this file,
if the file is modified or not, filesystem names, and so on.

\verb+YZSession+

A session is basically a client application instance, from the core point
of view. A session can have one or more files opened, and one or more views,
or "windows" opened. The session is responsible for creating and handling
both views and buffers (see \verb+createView()+, \verb+createBufferAndView()+,
\verb+createBuffer()+).

\verb+YZView+

A view is a GUI component displaying part of a file, within a session. As
such, it depends on both~: the constructor needs both a
\verb+YZSession+ and a \verb+YZBuffer+. This is the class where rendering is
done, closely with the core.

\section{Data paths}
In this section, we describe the paths followed by the data (file,
keyboard) within \emph{Yzis} architecture.

During initialisation of the program, an instance of \verb+YZSession+ is
created (or a subclass, most probably). This instance is unique, and is
always present. Most of high level API is here (like creating/deleting other
important objects).
For example, if you want to open a file, and create an associated view, you
can use
\begin{verbatim}
session->createBufferAndView( myname);
\end{verbatim}

At every time, the session has a 'current view', which is the one receiving
user input. For example in nyzis, if you haven't use the \emph{spilt}
feature, there's only one view displayed at any time, even if several
files are opened.  This is the current view. In Qyzis or Kyzis, on the
other way, several views are displayed, but only one window has the focus :
this is the current view from \verb+YZSession+ point of view.

The \emph{client} handles the event loop. When a key event is received, it is sent
to the \verb+libyzis+ core using the current view. For example in
\verb+nyzis/nsession.cpp+ we can read :
\begin{verbatim}
currentView()->sendKey( QString( QChar( c ) ), modifiers );
currentView()->sendKey( "<CTRL>]" );
\end{verbatim}

The core will do all what is needed : store the character in appropriate
buffer, update all views for this buffer, if the character is \emph{enter},
then some action might be taken, such as executing a command or whatever,
depending on the current mode.

Yzis core will use the API to call the upper GUI layer to display. (details
are in section \ref{rendering}).



\chapter{Components description}

\section{Modes}


\section{Rendering}
\label{rendering}

Loic ?

plan : global overview, messages exchanged between core and client.

TODO : interval, cursors, color, highlighting.

\section{Unit testing}

Phil ?

\section{Debugging and logging}

Phil ?

you can use the following macros
\begin{verbatim}
YZASSERT_MSG( assertion, msg )
YZASSERT( assertion )
YZASSERT_EQUALS( a, b )
\end{verbatim}

If you want to write some debugging message, please use the following
stream. You can add an optional (text) argument to specify a debug area.
\begin{verbatim}
yzDebug() yzWarning() yzError() yzFatal()
\end{verbatim}
Example~:
\begin{verbatim}
yzDebug("rendering") << "my message" << endl;
\end{verbatim}

Most clients are graphical GUIs, and as such,
they can use the \verb+stdout+ or \verb+stderr+ to print debugging
messagse. But nyzis is using the console for user interaction, and this
is not possible. This is why yzis debug messages are written in a file such
as \verb+/tmp/yzisdebug-orzel.log+. To see them, I recommand the following
command (use an alias) :
\begin{verbatim}
tail --follow=name /tmp/yzisdebug-orzel.log ,
\end{verbatim}
that you start from another console.

\section{Scripting}

who ?

\chapter{nyzis, ncurses client}

Thomas.

\chapter{qyzis, Qt-only client}

Thomas.

\chapter{Developer FAQ}

What's the mess with symbol \verb+scroll+ ?

The ncurses library (only used by nyzis) defines a MACRO with such a name.
This creates big clashes with Qt API as some classes (QWidget for example)
have methods with this name. Nyzis handles this using some hack (see
viewWidget.h), but to ease handling this, the word \verb+scroll+ should
just not be used anywhere in libyzis.


\end{document}

